\documentclass[12pt]{article}
\usepackage{geometry}
\geometry{
	a4paper,
	margin=2cm,
	left=3cm, bottom=2.5cm}

\usepackage[utf8]{inputenc}
\usepackage[german]{babel}
\usepackage[T1]{fontenc}
\usepackage{hyperref}
\usepackage{listings}
\usepackage{graphicx}
\usepackage{color}
\usepackage[toc]{glossaries}
\usepackage{chngcntr} %change counter
\usepackage{tabularx}
\usepackage{fancyhdr}
\usepackage{ulem}

%predefined colors: black, blue, cyan, green, magenta, red, white, yellow
\definecolor{commentgreen}{rgb}{0,0.6,0}
\definecolor{keywordblue}{rgb}{0,0,0.8}
\definecolor{orange}{rgb}{0.9,0.5,0}
\definecolor{internal_link}{rgb}{0,0,0.4}

\hypersetup{
	colorlinks=true, 
	linkcolor=internal_link, 
	urlcolor=cyan}

\lstset{
	language=C,
	basicstyle=\footnotesize\ttfamily,
	showstringspaces=false,
	commentstyle=\itshape\color{commentgreen},
	keywordstyle=\bfseries\color{keywordblue},
%	identifierstyle=\color{red},
	stringstyle=\color{orange}
}

\setlength{\parindent}{0em} %setzt Paragraphen-Einzug auf 0
\renewcommand{\labelitemi}{\textbullet}
\renewcommand\labelitemii{\textbullet}
\renewcommand\labelitemiii{\textbullet}
\renewcommand{\theenumi}{\arabic{enumi}}
\renewcommand{\theenumii}{\arabic{enumii}}
\renewcommand{\theenumiii}{\arabic{enumiii}}

\title{\textbf{Projektbegleitender Bericht}}
\author{"Interaktives TicTacToe-Spiel für Kinder"}
\date{\today} %richtige Formatierung mittels package babel (s.o.)



\begin{document}
\begin{titlepage}
	\centering	
	\begin{figure}
	\centering
	{\scshape\LARGE -INTERNES DOKUMENT- \par}
	\vspace{1cm}
	\end{figure}
	\maketitle
	\thispagestyle{empty}
	
	\vspace*{10cm}	
	

\end{titlepage}
	
\tableofcontents 
\newpage	
%"` links, "' rechts

\setcounter{page}{3}

\section{Vorabschätzung}
Dieser Abschnitt enthält vorab geschätzte Zeiten der einzelnen Teilabschnitte.
\\\\
\begin{tabularx}{\textwidth}{|X|X|} \hline
\textbf{Datum}&\textbf{Meilenstein}\\ \hline
18.03.2019 & Vorbereitung\\ \hline
29.03.2019 & Projektplan und Pflichtenheft\\ \hline
12.04.2019 & Analyse und Entwurf\\ \hline
03.05.2019 & Prototyp\\ \hline
10.05.2019 & Funktionstest\\ \hline
30.08.2019 & Release\\ \hline
13.09.2019 & Auslieferung\\ \hline
\end{tabularx}\\\\

\newpage

\section{Protokollierung}

\subsection{Kickoff Meeting}
Datum: 09.01.2019 \\ \\
\uline{Besprechung der Projektmodalitäten}
\begin{itemize}\itemsep0em
\item Teambildung
\item Dokumente in LaTeX\\
\end{itemize}

\uline{Abstecken grober Zielstellungen/Zeitrahmen}
\begin{itemize}\itemsep0em
\item Fertigstellung finales Produkt: 27.09.2019
\item Terminfestlegung für Grobplanung: 15.01.2019,\\
\textbf{bis dahin:} alle Mitarbeiter lesen sich in das Lastenheft ein\\
\end{itemize}

% wer hat wann, in welcher rolle wie viel und was gemacht \\


Fortschritt:\\\\
\begin{tabularx}{\textwidth}{|X|r|r|} \hline
\textbf{Gegenstand}&\textbf{Fertigstellung} & \textbf{in Prozent}\\ \hline
lines of code & xx/xx  & \\ \hline
interne Dokumente & xx/xx &  \\ \hline
externe Dokumente & xx/xx & \\ \hline
Entwicklungsumgebungen vorbereitet & xx/xx & \\ \hline
\textbf{Summe} & xx/xx &  \\ \hline
\end{tabularx}\\\\
\newpage

\subsection{Grobplanung}
Datum: 15.01.2019 \\

\uline{Verteilung der Aufgaben:}
\begin{itemize}\itemsep0em
\item Kalkulation: Siedler, Robert - Finanzabteilung
\item Überarbeitung Firmenporträt, Angebot: Günther, Emanuel - Entwickler
\item Pflichtenheft: Leopold, Michael - Geschäftsleitung
\item projektbegleitender Bericht: Creutzburg, Patrick - Werksstudent \\
\item Besprechen der Einarbeitung in das Lastenheft
\item Installation der Textentwicklungsumgebung Texmaker, Anfertigen einer Dokumentenvorlage
\item Anpassung des Corporate Designs durch Herrn Siedler\\
\end{itemize}

\uline{Zielstellung für das nächste Meeting:}
\begin{itemize}\itemsep0em
\item Ausfertigung der Pflichtenhefts
\item Angebot und interne Kalkulation
\item Anfertigen einer internen Dokumentation \\
\end{itemize}
%Michael: 10.02.2019 Homeoffice in der 2 KW
%Weitere Befuellung des Pflichtenhefts \\
%14.02.2019 Homeoffice in der 3 KW
%Weitere Befuellung des Pflichtenhefts \\

%Robert: \\
%Beginn Kalkulation, Design

%Alle: \\
%Micha: Pflichtenheft, Firmenportaet
%Robert: Corporate Design, Kalkulation
%Emanuel: Angebot
%Patirck: Dokumentation

%Ueberlegung wie das Programm Aufgebaut ist. --> %FinalStateMachine Grobfassung

%Schrieben des ersten Auszugs des Firmenporteats \\\\

Fortschritt:\\\\
\begin{tabularx}{\textwidth}{|X|r|r|} \hline
\textbf{Gegenstand}&\textbf{Fertigstellung} & \textbf{in Prozent}\\ \hline
lines of code & 0/300  & 0/100\\ \hline
interne Dokumente & 0/2 & 0/100  \\ \hline
externe Dokumente & 0/3 & 0/100 \\ \hline
Entwicklungsumgebungen vorbereitet & 1/3 & 33/100 \\ \hline
\textbf{Summe} & - & 8/100  \\ \hline
\end{tabularx}\\\\
\newpage

\subsection{Feinplanung}
Datum: 23.01.2019 \\

\uline{Fertiggestellt:}
\begin{itemize}\itemsep0em
\item Kalkulation (Herr Siedler, 4h)
\item Pflichtenheft, anteilig (Herr Leopold, 3h)
\item Angebot (Herr Günther, 2h)\\
\end{itemize}

\uline{Besprechnung:}
\begin{itemize}\itemsep0em
\item Vorstellung der Eintragungen vom 10.01. und 14.01. in das Pflichtenheft
\item Auswertung/Bewertung des bearbeiteten Corporate Designs
\item Auswertung der Kalkulation
\item Angebot: Abgleich mit Kalkulation \\
\end{itemize}
%Alle: \\
%kontrolle des Angebots und Firmenportaet. Beides fuer gut und %korrekt empfunden

%Welche Diagramme brauchen wir? --> Klassendiagramm(Vorgegeben) %Anwendungsfalldiagramm(Vorgegeben) Zustandsdiagramm 
%Diagramme Erstellt

%Erstellen des Ordners: Dokumentationen mit Gruppentreffen und %Personen seperiert
%Schrieben der ersten Inhalte der Dokumentation \\\\

Fortschritt:\\\\
\begin{tabularx}{\textwidth}{|X|r|r|} \hline
\textbf{Gegenstand}&\textbf{Fertigstellung} & \textbf{in Prozent}\\ \hline
lines of code & 0/300  & 0/100\\ \hline
interne Dokumente & 1/1 & 100/100  \\ \hline
externe Dokumente & 1/3 & 33/100 \\ \hline
Entwicklungsumgebungen vorbereitet & 1/3 & 33/100 \\ \hline
\textbf{Summe} & - & 42/100  \\ \hline
\end{tabularx}\\\\
\newpage

\subsection{Fertigstellung Planung}
Datum: 29.01.2019 \\

\uline{Fertiggestellt:}
\begin{itemize}\itemsep0em
\item Pflichtenheft (Herr Leopold, 4h)
\item Firmenporträt (Herr Günther, 2h)
\end{itemize}

\uline{Besprechung:}
\begin{itemize}\itemsep0em
\item Vorstellung Änderungen Pflichtenheft und Finalisierung
\item Vorstellung Änderungen Firmenporträt und Finalisierung
\end{itemize}

Fortschritt:\\\\
\begin{tabularx}{\textwidth}{|X|r|r|} \hline
\textbf{Gegenstand}&\textbf{Fertigstellung} & \textbf{in Prozent}\\ \hline
lines of code & 0/300  & 0/100\\ \hline
interne Dokumente & 1/1 & 100/100  \\ \hline
externe Dokumente & 3/3 & 100/100 \\ \hline
Entwicklungsumgebungen vorbereitet & 1/3 & 33/100 \\ \hline
\textbf{Summe} & - & 58/100  \\ \hline
\end{tabularx}\\\\

\newpage
\subsection{Aufsetzen des Repository}
Datum: 10.04.2019 \\

\uline{Fertiggestellt:}
\begin{itemize}\itemsep0em
\item Git Repo eingerichtet (Herr Günther,Leopold Michael, 1h)
\item Auswahl des Prototyping tool(Herr Creutzburg,1h)
\item Erste Ideen zur Prototyp Oberfläche(Herr Creutzburg,1h)
\end{itemize}

\uline{Besprechung:}
\begin{itemize}\itemsep0em
\item Programmierung, Grafiken, Projektbegleitender Bericht
\end{itemize}

Fortschritt:\\\\
\begin{tabularx}{\textwidth}{|X|r|r|} \hline
\textbf{Gegenstand}&\textbf{Fertigstellung} & \textbf{in Prozent}\\ \hline
lines of code & 0/300  & 0/100\\ \hline
interne Dokumente & 1/1 & 100/100  \\ \hline
externe Dokumente & 3/3 & 100/100 \\ \hline
Entwicklungsumgebungen vorbereitet & 1/3 & 33/100 \\ \hline
\textbf{Summe} & - & 58/100  \\ \hline
\end{tabularx}\\\\

\newpage


\newpage
\subsection{Implementierungsbesprechung}
Datum: 17.04.2019 \\

\uline{Fertiggestellt:}
\begin{itemize}\itemsep0em
\item Prototyp(Herr Creuzburg, 2h)
\item Umstrukturierung des Repositorys(Herr Leopold,Herr Günther, 1h)
\end{itemize}
\uline{Besprechung:}
\begin{itemize}\itemsep0em
\item Einigung auf Grafiken basiertes Design
\item Besprechung über die ersten zu nutzenden Grafiken
\item Texte für Copyright, Spielregeln etc. erstellen 
\end{itemize}

Fortschritt:\\\\
\begin{tabularx}{\textwidth}{|X|r|r|} \hline
\textbf{Gegenstand}&\textbf{Fertigstellung} & \textbf{in Prozent}\\ \hline
lines of code & 280/500  & 28/100\\ \hline
interne Dokumente & 10/10 & 100/100  \\ \hline
externe Dokumente & 4/4 & 100/100 \\ \hline
Entwicklungsumgebungen vorbereitet & 3/3 & 100/100 \\ \hline % bold hline
Grafiken des Spiels vorbereitet & 0/25 & 0/100 \\ \hline
\textbf{Summe} & - & 58/100  \\ \hline
\end{tabularx}\\\\
\newpage

\newpage
\subsection{Weiterführende Implementierungsgespräche}
Datum: 24.04.2019 \\

\uline{Besprechung:}
\begin{itemize}\itemsep0em
\item Grafiken des Spiels
\item Termine festlegen für weitere Aufgaben
\end{itemize}

Fortschritt:\\\\
\begin{tabularx}{\textwidth}{|X|r|r|} \hline
\textbf{Gegenstand}&\textbf{Fertigstellung} & \textbf{in Prozent}\\ \hline
lines of code & 280/500  & 56/100\\ \hline
interne Dokumente & 1/1 & 100/100  \\ \hline
externe Dokumente & 3/3 & 100/100 \\ \hline
Entwicklungsumgebungen vorbereitet & 1/3 & 33/100 \\ \hline
\textbf{Summe} & - & 58/100  \\ \hline
\end{tabularx}\\\\

\newpage


\newpage
\subsection{Bearbeitung Grundlegender offener Aufgaben}
Datum: 01.05.2019 \\

\uline{Fertiggestellt:}
\begin{itemize}\itemsep0em
\item Beginn Umbau auf Grafiken(Herr Günther, 4h)
\item Fortschrittstabelle und Pflichtenheft aktualisiert(Herr Leopold, 2,5h)
\end{itemize}

\uline{Besprechung:}
\begin{itemize}\itemsep0em
\item Folgende Aufgabenverteilung: \\
JavaDoc, JUnit, Code Styleguide(Herr Günther)\\Anwenderdokumentation(Herr Siedler)\\ Gradle,Pflichtenheft,projektbegleitender Bericht(Herr Leopold, 3h\\
\end{itemize}

Fortschritt:\\\\
\begin{tabularx}{\textwidth}{|X|r|r|} \hline
\textbf{Gegenstand}&\textbf{Fertigstellung} & \textbf{in Prozent}\\ \hline
lines of code & 325/500  & 65/100\\ \hline
interne Dokumente & 1/1 & 100/100  \\ \hline
externe Dokumente & 3/3 & 100/100 \\ \hline
Entwicklungsumgebungen vorbereitet & 1/3 & 33/100 \\ \hline
\textbf{Summe} & - & 58/100  \\ \hline
\end{tabularx}\\\\

\newpage

\newpage
\subsection{Arbeit an Dokumentation}
Datum: 8.05.2019 \\

\uline{Fertiggestellt:}
\begin{itemize}\itemsep0em
\item Screenshots fertig (Herr Siedler, Herr Creutzburg 1h)
\item finale Ertsellung der Grafiken(Herr Siedler,3h)
\item Pflichtenheft fertig mit Screenshots und Fortschrittstabelle(1h)
\item finale Grafiken eingepflegt(Herr Günther, 1h)
\item UML-Diagramme aktualisiert (Herr Günther, Herr Creutzburg 1h)
\item  manueller Test Programm, Änderungen an Grafik(alle,2h)\\
\end{itemize}

\uline{Besprechung:}
\begin{itemize}\itemsep0em
\item Gradle Probleme mit Integration beheben
\item Entscheidung: Änderung Pflichtenheft: Auflösung muss FullHD sein
\end{itemize}

Fortschritt:\\\\
\begin{tabularx}{\textwidth}{|X|r|r|} \hline
\textbf{Gegenstand}&\textbf{Fertigstellung} & \textbf{in Prozent}\\ \hline
lines of code & 468/468  & 100/100\\ \hline
interne Dokumente & 1/1 & 100/100  \\ \hline
externe Dokumente & 3/3 & 100/100 \\ \hline
Entwicklungsumgebungen vorbereitet & 1/3 & 33/100 \\ \hline
\textbf{Summe} & - & 80/100  \\ \hline
\end{tabularx}\\\\


\newpage
\subsection{Vervollständigung Dokumentation, Buildtool}
Datum: 15.05.2019 \\

\uline{Fertiggestellt:}
\begin{itemize}\itemsep0em
\item JUnit Tests
\end{itemize}

\uline{Besprechung:}
\begin{itemize}\itemsep0em
\item Anwenderdoku gegengelesen, verbesserungen werden umgestzt.
\item 'Gradle Umsetzung hat sich als sehr schwer herausgestellt!' --> Umstieg auf Ant

\end{itemize}

Fortschritt:\\\\
\begin{tabularx}{\textwidth}{|X|r|r|} \hline
\textbf{Gegenstand}&\textbf{Fertigstellung} & \textbf{in Prozent}\\ \hline
lines of code & 468/468  & 100/100\\ \hline
interne Dokumente & 1/1 & 100/100  \\ \hline
externe Dokumente & 3/3 & 100/100 \\ \hline
Entwicklungsumgebungen vorbereitet & 2/3 & 66/100 \\ \hline
\textbf{Summe} & - & 80/100  \\ \hline
\end{tabularx}\\\\

\newpage
\subsection{Vervollständigung Dokumentation, Buildtool}
Datum: 22.05.2019 \\

\uline{Fertiggestellt:}
\begin{itemize}\itemsep0em
\item JavaDoc,Google Style Guide, Ant(Herr Günther, )
\item Anwenderdoku (Herr Günther,Herr Creuztburg, 2,5h)
\end{itemize}

\uline{Besprechung:}
\begin{itemize}\itemsep0em
\item Präsentation
\end{itemize}

Fortschritt:\\\\
\begin{tabularx}{\textwidth}{|X|r|r|} \hline
\textbf{Gegenstand}&\textbf{Fertigstellung} & \textbf{in Prozent}\\ \hline
lines of code & 468/468  & 100/100\\ \hline
interne Dokumente & 1/1 & 100/100  \\ \hline
externe Dokumente & 3/3 & 100/100 \\ \hline
Entwicklungsumgebungen vorbereitet & 2/3 & 66/100 \\ \hline
\textbf{Summe} & - & 80/100  \\ \hline
\end{tabularx}\\\\


\clearpage

\end{document}